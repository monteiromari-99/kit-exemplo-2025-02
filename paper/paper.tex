% Documento LaTeX com o artigo que estamos esrevendo

%Cabeçalho 
% Onde configura o documento
%%%%%%%%%%%%%%%%%%%%%%%%%%%%%%%%%%%%%%%%%%%%%%%%%%%%%
\documentclass{article} %a classe do documento é um artigo

\usepackage[brazil]{babel}
\usepackage{graphicx}






%Corpo
% Onde a gente escreve o texto
%%%%%%%%%%%%%%%%%%%%%%%%%%%%%%%%%%%%%%%%%%%%%%%%%%%%%
% ambiente: está entre o begin e o end
\begin{document}

\title{Análise de variação de temperatura dos últimos cinco anos}
\author{Mariana Monteiro e Silva}
\maketitle



\begin{abstract}
ABSTRACTCHY

\end{abstract}

\section{Introdução}
\label{sec:intro}
%O primeiro paragrafo ele não identa e o segundo sim
%para iniciar outro parágrafo tem q deixar uma linha em branco
Isso aqui é a intro.
Another one

And anotha one

\section{Metodologia}
\label{sec:metodos}

A receita do bolo

Ajustamos uma reta aos cinco últimos snos dos dados
de temperatura média mensal para cada país.
Assim calculamos a taxa de variação da temperatura recente.

A equação da reta é

\begin{equation}
T(t) = at + b,
\label{eq:reta}
\end{equation}

\noindent
onde $T$ é a temperatura, $t$ é o tempo, $a$ é o coeficiente angular 
e $b$ é o coeficiente linear.

Utilizamos a equação \ref{eq:reta} em um código Python para fazer 
o ajuste da reta com o método dos mínimos quadrados. Isso está referenciado na seção \ref{sec:metodos}

\section{Resultados}
\label{sec:resultados}

Analisamos os dados de 225 países.

\begin{figure}
	\centering
	\includegraphics[width=0.5\columnwidth]{../figuras/variacao_temperatura.png}
	\caption{
		Variação de temperatura média mensal dos cinco últimos anos.
		a) Países com as cinco menores variações de temperatura.
		b) Países com as cinco maiores variações de temperatura.
	}
\label{fig:variacao}
\end{figure}


Os resultados da análise de variação de temperatura estão na figura \ref{fig:variacao}




















\end{document}